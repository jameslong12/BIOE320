\documentclass{article}

\usepackage{hyperref}
\hypersetup{
	colorlinks=true,
	linkcolor=blue,
	urlcolor=cyan,}
\usepackage{booktabs}
\usepackage{textgreek}

\input{../structure.tex} % Include the file specifying the document structure and custom commands

%----------------------------------------------------------------------------------------
%	ASSIGNMENT INFORMATION
%----------------------------------------------------------------------------------------

\title{Week 5 In-Lab}
\author{BIOE 320 Systems Physiology Laboratory} 
\date{}
%----------------------------------------------------------------------------------------

\begin{document}
\large
\maketitle

\textbf{Student Name:}\hfill 	\textbf{Total Grade:\ \ \ \ /25}\vspace{0.5cm}

\textbf{Student Name:}\hfill 	\textbf{Total Grade:\ \ \ \ /25}\vspace{0.5cm}

\textbf{Student Name:}\hfill 	\textbf{Total Grade:\ \ \ \ /25}\\

\section*{Setting Up}
\subsection*{Changing Sampling Rate}
\begin{itemize}
	\item[5.] What are the differences between the recordings at 20 Hz and 200 Hz?\vspace{3cm}
	\item[6.] What criteria should be used to establish an appropriate sampling rate?\vspace{3cm}
\end{itemize}

\section*{Data Analysis}
\subsection*{Segment 1}
\begin{itemize}
	\begin{table}[h]
	\centering
	\caption{Time and beats per minute (BPM) for three cardiac cycles}
	\begin{tabular}[h!]{p{0.15\linewidth}|p{0.15\linewidth}p{0.15\linewidth}p{0.15\linewidth}p{0.15\linewidth}}
	\toprule
	Measurement & Cycle 1 & Cycle 2 & Cycle 3 & Mean\\
	\midrule
	\textDelta time (s) & & & &\\& & & &\\
	BPM & & & &\\& & & &\\
	\bottomrule
	\end{tabular}
	\end{table}

	\begin{table}[h]
	\centering
	\caption{Characteristics of the ECG for cardiac cycle 1}
	\begin{tabular}[h!]{p{0.3\linewidth}|p{0.25\linewidth}p{0.25\linewidth}}
	\toprule
	ECG trace & \textDelta time (sec) & \textDelta amplitude (mV)\\
	\midrule
	P wave & & \\& & \\& & \\
	PR interval & & \\& & \\& & \\
	QRS complex & & \\& & \\& & \\
	T wave & & \\& & \\& & \\
	Q wave to end of T wave & & \\(\textit{ventricular systole})& & \\& & \\
	End of T wave to end of P wave & & \\(\textit{ventricular diastole})& & \\& & \\
	Peak of P wave to end P wave & & \\(\textit{ventricular diastole})& & \\& & \\
	\bottomrule
	\end{tabular}
	\end{table}
	
	\item[3.] Is there always one P wave for every QRS complex? If not, what would this signify?\vspace{2.5cm}
	\item[4.] Compare and contrast the shape (duration and amplitude) of the P and T waves. Give the mechanical and electrical reasons for the differences.\vspace{3.5cm}
\end{itemize}

\subsection*{Segment 2}
\begin{table}[h]
	\centering
	\caption{Time and beats per minute (BPM) while sitting}
	\begin{tabular}[h!]{p{0.15\linewidth}|p{0.15\linewidth}p{0.15\linewidth}p{0.15\linewidth}p{0.15\linewidth}}
	\toprule
	Measurement & Cycle 1 & Cycle 2 & Cycle 3 & Mean\\
	\midrule
	\textDelta time (s) & & & &\\& & & &\\
	BPM & & & &\\& & & &\\
	\bottomrule
	\end{tabular}
	\end{table}
	
\begin{itemize}
	\item[2.] Explain the observed heart rate variations in sitting up vs. supine positioning. Describe the physiological mechanisms causing these differences.\vspace{3.5cm}
\end{itemize}

\subsection*{Segment 3}
\begin{table}[h]
	\centering
	\caption{Time and beats per minute (BPM) while deep breathing}
	\begin{tabular}[h!]{p{0.15\linewidth}|p{0.15\linewidth}p{0.15\linewidth}p{0.15\linewidth}p{0.15\linewidth}}
	\toprule
	Measurement & Cycle 1 & Cycle 2 & Cycle 3 & Mean\\
	\midrule
	\textit{Inspiration} & & & &\\
	\textDelta time (s) & & & &\\& & & &\\
	BPM & & & &\\& & & &\\
	\textit{Expiration} & & & &\\
	\textDelta time (s) & & & &\\& & & &\\
	BPM & & & &\\& & & &\\
	\bottomrule
	\end{tabular}
	\end{table}
	
\begin{itemize}
	\item[2.] Are there differences in the cardiac cycle with the respiratory cycle (inspiration vs. expiration)? If so, what is the physiological basis for these differences?\pagebreak
\end{itemize}

\subsection*{Segment 4}
\begin{table}[h]
	\centering
	\caption{Time and beats per minute (BPM) after exercising}
	\begin{tabular}[h!]{p{0.18\linewidth}|p{0.15\linewidth}p{0.15\linewidth}p{0.15\linewidth}p{0.15\linewidth}}
	\toprule
	Measurement & Cycle 1 & Cycle 2 & Cycle 3 & Mean\\
	\midrule
	\textit{Start of recording} & & & &\\
	\textDelta time (s) & & & &\\& & & &\\
	BPM & & & &\\& & & &\\
	\textit{End of recording} & & & &\\
	\textDelta time (s) & & & &\\& & & &\\
	BPM & & & &\\& & & &\\
	\bottomrule
	\end{tabular}
	\end{table}\vspace{1cm}
	
\begin{table}[h]
	\centering
	\caption{Characteristics of the ECG after exercising}
	\begin{tabular}[h!]{p{0.3\linewidth}|p{0.25\linewidth}}
	\toprule
	ECG trace & \textDelta time (sec)\\
	\midrule
	Q wave to end of T wave & \\(\textit{ventricular systole})& \\& \\
	End of T wave to end of P wave & \\(\textit{ventricular diastole})& \\& \\
	Peak of P wave to end P wave & \\(\textit{ventricular diastole})& \\& \\
	\bottomrule
	\end{tabular}
	\end{table}
	
\begin{itemize}
	\item[2.] What changes occurred in the duration of systole and diastole between resting (Table 2) and immediately after exercise? What could account for these changes?
\end{itemize}

\end{document}
