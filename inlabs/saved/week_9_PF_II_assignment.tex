\documentclass{article}

\usepackage{hyperref}
\hypersetup{
	colorlinks=true,
	linkcolor=blue,
	urlcolor=cyan,}
\usepackage{booktabs}
\usepackage{textgreek}
\usepackage{gensymb}

\input{../structure.tex} % Include the file specifying the document structure and custom commands

%----------------------------------------------------------------------------------------
%	ASSIGNMENT INFORMATION
%----------------------------------------------------------------------------------------

\title{Week 9 In-Lab}
\author{BIOE 320 Systems Physiology Laboratory} 
\date{}
%----------------------------------------------------------------------------------------

\begin{document}
\large
\maketitle

\begin{itemize}
	\begin{table}[h]
	\centering
	\caption{Measures of pulmonary function at different times after breathing start}
	\begin{tabular}[h!]{p{0.3\linewidth}p{0.12\linewidth}p{0.12\linewidth}p{0.12\linewidth}p{0.12\linewidth}}
	\toprule
	Condition & 0 min & 1 min & 2 min & 3 min\\
	\midrule
	Normal O\textsubscript{2} (\%) & & & & N/A\\\midrule
	Normal CO\textsubscript{2} (\%) & & & & N/A\\\midrule
	Normal VO\textsubscript{2} (L) & & & & N/A\\\midrule
	Exercise O\textsubscript{2} (\%) & & & & N/A\\\midrule
	Exercise CO\textsubscript{2} (\%) & & & & N/A\\\midrule
	Exercise VO\textsubscript{2} (L) & & & & N/A\\\midrule
	Hyperventilation O\textsubscript{2} (\%) & & & & \\\midrule
	Hyperventilation CO\textsubscript{2} (\%) & & & & \\\midrule
	Hyperventilation VO\textsubscript{2} (L) & & & & \\
	\bottomrule
	\end{tabular}
	\end{table}\vspace{0cm}
	
	\begin{table}[h]
	\centering
	\caption{Molar concentrations of O\textsubscript{2} in various conditions}
	\begin{tabular}[h!]{p{0.35\linewidth}p{0.2\linewidth}p{0.2\linewidth}}
	\toprule
	Condition & O\textsubscript{2} (\%) & [O\textsubscript{2}] (M or mM)\\\midrule
	A: Baseline in chamber & & \\\midrule
	B: Normal breathing at 2 min & & \\\midrule
	C: Exercise at 1 min & & \\\midrule
	D: Before hyperventilation at 1 min & & \\\midrule
	E: After hyperventilation at 2 min & & \\
	\bottomrule
	\end{tabular}
	\end{table}\vspace{0cm}
	
	\item[3.] Compare and explain the results. Focus on comparisons between A \& B, B \& C, D \& E, and B \& E.\pagebreak
	
	\begin{table}[h]
	\centering
	\caption{Breathing rate and tidal volume over three conditions}
	\begin{tabular}[h!]{p{0.3\linewidth}p{0.3\linewidth}p{0.3\linewidth}}
	\toprule
	Condition & Breathing rate (breaths/min) & Tidal volume (L)\\\midrule
	Normal & & \\\midrule
	Exercise & & \\\midrule
	During hyperventilation & & \\
	\bottomrule
	\end{tabular}
	\end{table}\vspace{0cm}
	
	\item[5.] Explain your method for determining the breathing rate and the tidal volume.\vspace{3cm}
	\item[6.] Based on your data, how many moles of O\textsubscript{2} are consumed per breath for normal breathing? How many moles of CO\textsubscript{2} are produced per breath for normal breathing? The volume of the mixing chamber is 5 L. Record these numbers for the post-lab.\vspace{5cm}
	\item[7.] Calculate the pulmonary ventilation rate for each of the three conditions through the following steps:
	\begin{table}[h]
	\centering
	\caption{Pulmonary ventilation rate over three conditions}
	\begin{tabular}[h!]{p{0.3\linewidth}p{0.32\linewidth}p{0.32\linewidth}}
	\toprule
	Condition & Minute respiratory rate (L/min) & Alveolar ventilation rate (L/min)\\\midrule
	Normal & & \\\midrule
	Exercise & & \\\midrule
	During hyperventilation & & \\
	\bottomrule
	\end{tabular}
	\end{table}\vspace{0cm}
	
	\begin{itemize}
	\item[(a)] Calculate the minute respiratory rate (total pulmonary ventilation rate) and record in the chart. Explain how you performed the calculation.\pagebreak
	\item[(b)] Compare and briefly explain differences in minute respiratory rate among the test conditions.\vspace{3cm}
	\item[(c)] Calculate the alveolar ventilation rate and record in the chart. Assume a dead space volume of 150 mL/breath. Explain how you performed the calculation.\vspace{3cm}
	\item[(d)] Compare and briefly explain differences in alveolar ventilation rate among the test conditions.\vspace{3cm}
	\item[(e)] Which of the two ventilation rates is a more accurate indicator of the efficiency of actual breathing/ventilation? Explain.\vspace{3cm}
	\end{itemize}
	
	\item[8.] The rate of oxygen consumption is equal to the rate of oxygen diffusion across the respiratory membrane. Determine the rate of oxygen consumption for the normal and exercise conditions based on your experimental data.\vspace{3cm}
	\item[9.] How do the measured O\textsubscript{2} consumption rates at rest and after exercise compare?
\end{itemize}
\end{document}
