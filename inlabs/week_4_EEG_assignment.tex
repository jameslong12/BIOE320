\documentclass{article}

\usepackage{hyperref}
\hypersetup{
	colorlinks=true,
	linkcolor=blue,
	urlcolor=cyan,}
\usepackage{booktabs}
\usepackage{textgreek}

\input{../structure.tex} % Include the file specifying the document structure and custom commands

%----------------------------------------------------------------------------------------
%	ASSIGNMENT INFORMATION
%----------------------------------------------------------------------------------------

\title{Week 4 In-Lab}
\author{BIOE 320 Systems Physiology Laboratory} 
\date{}
%----------------------------------------------------------------------------------------

\begin{document}
\large
\maketitle

\textbf{Student Name:}\hfill 	\textbf{Total Grade:\ \ \ \ /25}\vspace{0.5cm}

\textbf{Student Name:}\hfill 	\textbf{Total Grade:\ \ \ \ /25}\vspace{0.5cm}

\textbf{Student Name:}\hfill 	\textbf{Total Grade:\ \ \ \ /25}\\

\section*{Experiment 1}
\begin{itemize}

\begin{table}[h]
	\centering
	\caption{EEG amplitudes for each brain wave}
	\begin{tabular}[h!]{p{0.08\linewidth}|p{0.28\linewidth}p{0.28\linewidth}p{0.28\linewidth}}
	\toprule
	Rhythm  & Eyes Closed stddev (\textmu V) & Eyes Open stddev (\textmu V) & Eyes Re-closed stddev (\textmu V)\\
	\midrule
	Alpha & & &\\& & &\\
	\midrule
	Beta & & &\\& & &\\
	\midrule
	Delta & & &\\& & &\\
	\midrule
	Theta & & &\\& & &\\
	\bottomrule
	\end{tabular}
	\end{table}

	\item[4.] What information does using the computed value of the standard deviation of the amplitude give you as compared to other statistical measurements such as the mean of the amplitude?\pagebreak
	
	\begin{table}[h]
	\centering
	\caption{Frequency measurements for each brain wave}
	\begin{tabular}[h!]{p{0.08\linewidth}|p{0.22\linewidth}p{0.22\linewidth}p{0.22\linewidth}p{0.2\linewidth}}
	\toprule
	Rhythm  & Cycle 1 (Hz) & Cycle 2 (Hz) & Cycle 3 (Hz) & Mean (Hz)\\
	\midrule
	Alpha & & & &\\& & & &\\
	\midrule
	Beta & & & &\\& & & &\\
	\midrule
	Delta & & & &\\& & & &\\
	\midrule
	Theta & & & &\\& & & &\\
	\bottomrule
	\end{tabular}
	\end{table}
	
	\item[10.] Are the frequencies within the range of expected, published values?\vspace{1cm}
	\item[11.] Examine the alpha and beta waveforms for change between the "eyes closed" and "eyes open" states. Does desynchronization of the alpha rhythm occur when the eyes are open? Explain.\vspace{3cm}
	\item[12.] Does the beta rhythm become more pronounced in the "eyes open" state?\vspace{1cm}
	\item[13.] Examine the delta and theta rhythms. Is there an increase, decrease, or no change in the delta and theta activity when the eyes are open? Explain your observations.\vspace{4cm}
	\item[14.] Could you see evidence of synchrony and/or alpha block in your EEG measurements? Explain.
\end{itemize}
\pagebreak

\section*{Experiment 2}
\begin{table}[h]
	\centering
	\caption{Alpha wave characteristics under different conditions}
	\begin{tabular}[h!]{p{0.08\linewidth}|p{0.22\linewidth}p{0.22\linewidth}p{0.22\linewidth}}
	\toprule
	Segment  & Condition & stddev (\textmu V) & RMS (\textmu V)\\
	\midrule
	Alpha & & &\\& & &\\
	\midrule
	Beta & & &\\& & &\\
	\midrule
	Delta & & &\\& & &\\
	\midrule
	Theta & & &\\& & &\\
	\bottomrule
	\end{tabular}
	\end{table}

\begin{itemize}
	\item[4.] Under what conditions where the alpha wave amplitudes highest? Why?\vspace{3.5cm}
	\item[5.] How did the level of concentration (focused thinking) affect the data?\vspace{3.5cm}
	\item[6.] What kind of differences would you expect when measuring the amplitude of alpha and beta waves recorded from a subject tested alone in a darkened room and a subject tested in a lab full of students? Justify.\vspace{4cm}
\end{itemize}

\begin{info}
	Export data from Segment 1 (eyes open) and Segment 3 (hyperventilation). You will need these values to complete your post-lab assignment.
\end{info}
\end{document}
