\documentclass{article}

\usepackage{hyperref}
\hypersetup{
	colorlinks=true,
	linkcolor=blue,
	urlcolor=cyan,}
\usepackage{booktabs}
\usepackage{textgreek}

\input{../structure.tex} % Include the file specifying the document structure and custom commands

%----------------------------------------------------------------------------------------
%	ASSIGNMENT INFORMATION
%----------------------------------------------------------------------------------------

\title{Week 1 In-Lab}
\author{BIOE 320 Systems Physiology Laboratory} 
\date{}
%----------------------------------------------------------------------------------------

\begin{document}
\maketitle
\large

\section*{Data Viewing Tools}
\begin{itemize}
	\item[2.] Select the different markers and note from the markers what experiments were performed with this data set.
	
		\begin{itemize}
			\item[(a)]
			
			\item[(b)]\vspace{1cm}
			
			\item[(c)]\vspace{1cm}
		\end{itemize}

\end{itemize}

\section*{Channel Measurement Box}
\begin{itemize}		
	\item[2.] Using the first cycle (first beat), answer the following questions:
		\begin{itemize}
			\item[(a)] What is your measured value for Delta T?\vspace{1cm}
			\item[(b)] What does Delta T represent?\vspace{1cm}
			\item[(c)] What is your measured value for BPM?\vspace{1cm}
			\item[(d)] What does BPM represent?\vspace{1cm}
			\item[(e)] What are the "max" and "min" for the waveform on ECG?\vspace{1cm}
			\item[(g)] What do the "max" and "min" values represent?\vspace{2cm}
			\item[(h)] Try out the options for "area", "P-P", "integral", "stdev", and two additional ones. Describe a physiological example when each of these measurements may be useful.\vspace{9cm}
		\end{itemize}
\end{itemize}

\section*{Markers}
\begin{enumerate}
	\item How do you insert a marker while data are being recorded?\vspace{1.5cm}
	\item How do you insert text for a marker?\vspace{1.5cm}
	\item How do you insert an event marker after data have been recorded? Do this as practice.\vspace{2cm}
\end{enumerate}

\section*{Practice}
\begin{enumerate}
			\item Maximum values of:
				\begin{enumerate}
					\item First wave?\vspace{0.5cm}
					\item Second wave?\vspace{0.5cm}
					\item Third wave?\vspace{0.5cm}
					\item Fourth wave?\vspace{0.5cm}
				\end{enumerate}
			\item Frequency of single waveform?\vspace{1cm}
			\item Minimum and maximum amplitudes found between the first and fourth peak?\vspace{1cm}
			\item Interval of time between:
				\begin{enumerate}
					\item First and fourth peak?\vspace{0.5cm}
					\item Third and fourth peak?\vspace{0.5cm}
				\end{enumerate}
			\item Briefly describe the function of the following tools:
				\begin{enumerate}
					\item Selection tool\vspace{1cm}
					\item I-Beam tool\vspace{1cm}
					\item Zoom tool\vspace{1cm}
				\end{enumerate}
			\item You have lost track of where you are in the data due to zooming, scrolling, etc. List 2 sequential steps you can take to make sure the complete data file is on the screen.
		\end{enumerate}
\end{document}
