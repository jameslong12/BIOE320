\documentclass{article}

\usepackage{hyperref}
\hypersetup{
	colorlinks=true,
	linkcolor=blue,
	urlcolor=cyan,}
\usepackage{booktabs}
\usepackage{textgreek}

\input{../structure.tex} % Include the file specifying the document structure and custom commands

%----------------------------------------------------------------------------------------
%	ASSIGNMENT INFORMATION
%----------------------------------------------------------------------------------------

\title{Week 6 In-Lab}
\author{BIOE 320 Systems Physiology Laboratory} 
\date{}
%----------------------------------------------------------------------------------------

\begin{document}
\large
\maketitle

\subsection*{Data Analysis}
\begin{itemize}
	\begin{table}[h]
	\centering
	\caption{Mean R wave amplitudes for leads I and III in various conditions}
	\begin{tabular}[h!]{p{0.2\linewidth}|p{0.3\linewidth}p{0.3\linewidth}}
	\toprule
	Condition (n>8) & Mean amplitude (mV, lead I) & Mean amplitude (mV, lead I)\\
	\midrule
	Supine & & \\ & & \\
	\midrule
	Sitting Up & & \\ & & \\
	\midrule
	Inhaling & & \\ & & \\
	\midrule
	Exhaling & & \\ & & \\
	\bottomrule
	\end{tabular}
	\end{table}\vspace{1cm}
	
	\item[2.] Using data from the table, graphically determine the Mean Electrical Magnitude and Mean Electrical Axis for the conditions of Supine and Sitting Up.
	\begin{figure}[h]
	\centering\includegraphics[width=0.9\textwidth]{../images/ECG_II_7.jpg}
		\label{boxes}
		\end{figure}

	\item[3.] Explain the difference (if any) in the amplitudes of Leads I and III, as well as the Mean Electrical Magnitude and Axis under the two conditions.\vspace{5cm}
	\item[4.] Using data from the table, graphically determine the Mean Electrical Magnitude and Mean Electrical Axis for the conditions of Inhaling and Exhaling.
	\begin{figure}[h]
	\centering\includegraphics[width=0.9\textwidth]{../images/ECG_II_7.jpg}
		\label{boxes}
		\end{figure}
		
	\item[5.] Explain the difference (if any) in the amplitudes of Leads I and III, as well as the Mean Electrical Magnitude and Axis under the two conditions.\vspace{5cm}
	\item[6.] Give normal ranges of the mean electrical axis. Is your data within range?\vspace{3cm}
	\item[7.] What factors affect the orientation of the mean electrical axis (list at least 2)?\vspace{4cm}
	\item[8.] What factors affect the amplitude of the R wave recorded on the different leads (list at least 2)?\vspace{4cm}
	\item[9.] It is unlikely your ECG data looks exactly like "textbook" data. List at least 3 sources of experimental error.
\end{itemize}
\end{document}
