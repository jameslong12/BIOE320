\documentclass{article}

\usepackage{hyperref}
\hypersetup{
	colorlinks=true,
	linkcolor=blue,
	urlcolor=cyan,}
\usepackage{booktabs}
\usepackage{textgreek}

\input{../structure.tex} % Include the file specifying the document structure and custom commands

%----------------------------------------------------------------------------------------
%	ASSIGNMENT INFORMATION
%----------------------------------------------------------------------------------------

\title{Week 3 In-Lab}
\author{BIOE 320 Systems Physiology Laboratory} 
\date{}
%----------------------------------------------------------------------------------------

\begin{document}
\large
\maketitle

\textbf{Student Name:}\hfill 	\textbf{Total Grade:\ \ \ \ /25}\vspace{0.5cm}

\textbf{Student Name:}\hfill 	\textbf{Total Grade:\ \ \ \ /25}\vspace{0.5cm}

\textbf{Student Name:}\hfill 	\textbf{Total Grade:\ \ \ \ /25}\\

\section*{Ankle Jerk Reflex}
\begin{itemize}
	\begin{table}[h]
	\centering
	\caption{EMG measurements for dominant forearm}
	\begin{tabular}[h!]{p{0.08\linewidth}|p{0.25\linewidth}p{0.25\linewidth}p{0.25\linewidth}}
	\toprule
	Strike & Reaction time (ms) & Hammer activity (mV) & Muscle response (mV)\\
	\midrule
	1 & & &\\& & &\\
	\midrule
	2 & & &\\& & &\\
	\midrule
	3 & & &\\& & &\\
	\midrule
	4 & & &\\& & &\\
	\midrule
	5 & & &\\& & &\\
	\midrule
	6 & & &\\& & &\\
	\midrule
	7 & & &\\& & &\\
	\midrule
	8 & & &\\& & &\\
	\midrule
	9 & & &\\& & &\\
	\midrule
	10 & & &\\& & &\\
	\midrule
	$\mu \pm s$ & & &\\& & &\\
	\bottomrule
	\end{tabular}
	\end{table}
	\pagebreak
	\item[17.] Is there a significant difference between the reflex reaction time and the voluntary reaction time? Conduct a statistical test to determine if there is a statistically significant difference. Explain your procedure and calculations. Give a physiological explanation for your results.\vspace{6.7cm}
\end{itemize}

\section*{Knee Jerk Reflex}
\subsection*{Regular Procedure}
\begin{itemize}
	\begin{table}[h]
	\centering
	\caption{EMG measurements for dominant forearm}
	\begin{tabular}[h!]{p{0.08\linewidth}|p{0.25\linewidth}p{0.25\linewidth}p{0.25\linewidth}}
	\toprule
	Strike & Reaction time (ms) & Hammer activity (mV) & Muscle response (mV)\\
	\midrule
	1 & & &\\& & &\\
	\midrule
	2 & & &\\& & &\\
	\midrule
	3 & & &\\& & &\\
	\midrule
	4 & & &\\& & &\\
	\midrule
	5 & & &\\& & &\\
	\midrule
	6 & & &\\& & &\\
	\midrule
	7 & & &\\& & &\\
	\midrule
	8 & & &\\& & &\\
	\midrule
	9 & & &\\& & &\\
	\midrule
	10 & & &\\& & &\\
	\midrule
	$\mu \pm s$ & & &\\& & &\\
	\bottomrule
	\end{tabular}
	\end{table}
	\item[13.] Is there a relationship between hammer strike force and reaction time? Explain.\vspace{6cm}
	\item[14.] Is there are relationship between hammer strike force and the magnitude of the muscle response? Explain.\vspace{6cm}
	\item[15.] Measure the length of this reflex arc, realizing that it involves the L2, L3, and L4 segments of the spinal cord. Calculate an estimate of the nerve conduction velocity. Do you expect this to be an overestimate or an underestimate? Why? \textbf{Record the conduction velocity as you will need this information for your post-lab assignment}.\pagebreak
\end{itemize}

\subsection*{Jendrassik Maneuver}
\begin{itemize}
\begin{table}[h]
	\centering
	\caption{EMG measurements for dominant forearm}
	\begin{tabular}[h!]{p{0.08\linewidth}|p{0.25\linewidth}p{0.25\linewidth}p{0.25\linewidth}}
	\toprule
	Strike & Reaction time (ms) & Hammer activity (mV) & Muscle response (mV)\\
	\midrule
	1 & & &\\& & &\\
	\midrule
	2 & & &\\& & &\\
	\midrule
	3 & & &\\& & &\\
	\midrule
	4 & & &\\& & &\\
	\midrule
	5 & & &\\& & &\\
	\midrule
	6 & & &\\& & &\\
	\midrule
	7 & & &\\& & &\\
	\midrule
	8 & & &\\& & &\\
	\midrule
	9 & & &\\& & &\\
	\midrule
	10 & & &\\& & &\\
	\midrule
	$\mu \pm s$ & & &\\& & &\\
	\bottomrule
	\end{tabular}
	\end{table}
	\item[7.] How do the reaction times during the Jendrassik maneuver compare to the regular procedure? Explain the change or lack of change.\pagebreak
	\item[8.] How do the muscle response magnitudes during the Jendrassik maneuver compare to the regular procedure? Recall the negative feedback diagram (Fig. \ref{feedback}). Which variable do you think has changed?\vspace{8cm}
	\item[9.] One feature of negative feedback systems with high gain is \textit{ringing}, the presence of transient oscillations in the regulated variable before it settles to a steady-state value. Did you observe ringing in any of your experiments? If so, in what test(s)? Briefly describe how this occurs in terms of the feedback diagram. Why does high gain make ringing more likely?
\end{itemize}
\end{document}
