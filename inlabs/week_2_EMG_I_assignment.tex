\documentclass{article}

\usepackage{hyperref}
\hypersetup{
	colorlinks=true,
	linkcolor=blue,
	urlcolor=cyan,}
\usepackage{booktabs}
\usepackage{textgreek}

\input{../structure.tex} % Include the file specifying the document structure and custom commands

%----------------------------------------------------------------------------------------
%	ASSIGNMENT INFORMATION
%----------------------------------------------------------------------------------------

\title{Week 2 In-Lab}
\author{BIOE 320 Systems Physiology Laboratory} 
\date{}
%----------------------------------------------------------------------------------------

\begin{document}
\large
\maketitle

\textbf{Student Name:}\hfill 	\textbf{Total Grade:\ \ \ \ /25}\vspace{0.5cm}

\textbf{Student Name:}\hfill 	\textbf{Total Grade:\ \ \ \ /25}\vspace{0.5cm}

\textbf{Student Name:}\hfill 	\textbf{Total Grade:\ \ \ \ /25}\\

\section*{Data Analysis}
\begin{itemize}
	\item[6.]
	
	\begin{table}[h!]
	\centering
	\caption{EMG measurements for dominant forearm}
	\begin{tabular}[h!]{p{0.1\linewidth}|p{0.2\linewidth}p{0.2\linewidth}p{0.2\linewidth}p{0.2\linewidth}}
	\toprule
	Cluster & min (mV) & max (mV) & P-P (mV) & mean (mV-sec)\\
	\midrule
	1 & & & &\\& & & &\\
	
	\midrule
	2 & & & &\\& & & &\\
	\midrule
	3 & & & &\\& & & &\\
	\midrule
	4 & & & &\\& & & &\\
	\bottomrule
	\end{tabular}
	\end{table}
	
	\begin{itemize}
		\item[(a)] What is the percentage increase or decrease in EMG activity between the weakest and strongest clench for the dominant forearm? Show your calculations.\vspace{4cm}
		\item[(b)] Why can't you use the raw EMG signal to determine the mean value? Why must the integrated EMG signal be used?\vspace{3cm}
	\end{itemize}
	
	\item[7.]
	\begin{table}[h!]
	\centering
	\caption{EMG measurements for non-dominant forearm}
	\begin{tabular}[h!]{p{0.1\linewidth}|p{0.2\linewidth}p{0.2\linewidth}p{0.2\linewidth}p{0.2\linewidth}}
	\toprule
	Cluster & min (mV) & max (mV) & P-P (mV) & mean (mV-sec)\\
	\midrule
	1 & & & &\\& & & &\\
	
	\midrule
	2 & & & &\\& & & &\\
	\midrule
	3 & & & &\\& & & &\\
	\midrule
	4 & & & &\\& & & &\\
	\bottomrule
	\end{tabular}
	\end{table}
	
	\begin{itemize}
		\item[(a)] Compare the mean values of the strongest clench in EMG activity between the two forearms. Report the difference between the two forearms as a magnitude (mV or mV-sec) as well as a percentage (\%). Show your calculations.\vspace{6cm}
		\item[(b)] Does the dominant or non-dominant forearm show the highest EMG clench? Explain the physiological basis of your results.\vspace{5cm}
		\item[(c)] List four factors that influence maximum clench strength.\vspace{4cm}
	\end{itemize}
	
	\item[8.]
	\begin{table}[h!]
	\centering
	\caption{Tonus measurements}
	\begin{tabular}[h!]{p{0.1\linewidth}|p{0.2\linewidth}p{0.2\linewidth}|p{0.2\linewidth}p{0.2\linewidth}}
	\toprule
	& Forearm 1 & & Forearm 2 &\\
	Period & min (mV) & max (mV) & P-P (mV) & mean (mV-sec)\\
	\midrule
	1 & & & &\\& & & &\\
	
	\midrule
	2 & & & &\\& & & &\\
	\midrule
	3 & & & &\\& & & &\\
	\bottomrule
	\end{tabular}
	\end{table}
	
	\begin{itemize}
		\item[(a)] Is there a difference in tonus between the two forearms? If so, quantify (and show your work) and explain why.\vspace{6cm}
	\end{itemize}
\end{itemize}

\section*{Examining Other Motions}
\begin{itemize}
	\item[4.]
	\begin{table}[h!]
	\centering
	\caption{EMG measurements for other motions}
	\begin{tabular}[h!]{p{0.1\linewidth}|p{0.2\linewidth}p{0.2\linewidth}p{0.2\linewidth}p{0.2\linewidth}}
	\toprule
	Motion & min (mV) & max (mV) & P-P (mV) & mean (mV-sec)\\
	\midrule
	& & & &\\& & & &\\
	\midrule
	& & & &\\& & & &\\
	\midrule
	& & & &\\& & & &\\
	\bottomrule
	\end{tabular}
	\end{table}
	\begin{itemize}
		\item[(a)] Do you observe any differences among the EMG activity for the three tasks? If so, quantify the differences and explain the physiological origin of the difference. If not, explain why no difference should be observed.\vspace{4cm}
		\item[(b)] Predict the shape and magnitude of the EMG signal during the following scenarios. You may test these scenarios out if you want, using the protocol described previously.
		\begin{itemize}
			\item[i.] Grasping an object until your muscle is fatigued.\vspace{7cm}
			\item[ii.] Muscle undergoing isotonic contraction compared to isometric contraction.
		\end{itemize}
	\end{itemize}
\end{itemize}
\end{document}
