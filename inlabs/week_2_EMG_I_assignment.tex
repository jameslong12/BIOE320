\documentclass{article}

\usepackage{hyperref}
\hypersetup{
	colorlinks=true,
	linkcolor=blue,
	urlcolor=cyan,}
\usepackage{booktabs}
\usepackage{textgreek}

\input{../structure.tex} % Include the file specifying the document structure and custom commands

%----------------------------------------------------------------------------------------
%	ASSIGNMENT INFORMATION
%----------------------------------------------------------------------------------------

\title{Week 2 In-Lab}
\author{BIOE 320 Systems Physiology Laboratory} 
\date{}
%----------------------------------------------------------------------------------------

\begin{document}
\large
\maketitle


\section*{Data Analysis}
\begin{itemize}
	\item[6.]
	
	\begin{table}[h!]
	\centering
	\caption{EMG measurements for dominant forearm}
	\begin{tabular}[h!]{p{0.1\linewidth}|p{0.2\linewidth}p{0.2\linewidth}p{0.2\linewidth}p{0.2\linewidth}}
	\toprule
	Cluster & min (mV) & max (mV) & P-P (mV) & mean (mV-sec)\\
	\midrule
	1 & & & &\\& & & &\\
	
	\midrule
	2 & & & &\\& & & &\\
	\midrule
	3 & & & &\\& & & &\\
	\midrule
	4 & & & &\\& & & &\\
	\bottomrule
	\end{tabular}
	\end{table}
	
	\begin{itemize}
		\item[(a)] What is the percentage increase or decrease in EMG activity between the weakest and strongest clench for the dominant forearm? Show your calculations.\vspace{4cm}
		\item[(b)] Why can't you use the raw EMG signal to determine the mean value? Why must the integrated EMG signal be used?\pagebreak
	\end{itemize}
	
	\item[7.]
	\begin{table}[h!]
	\centering
	\caption{EMG measurements for non-dominant forearm}
	\begin{tabular}[h!]{p{0.1\linewidth}|p{0.2\linewidth}p{0.2\linewidth}p{0.2\linewidth}p{0.2\linewidth}}
	\toprule
	Cluster & Min (mV) & Max (mV) & P-P (mV) & Mean (mV-sec)\\
	\midrule
	1 & & & &\\& & & &\\
	
	\midrule
	2 & & & &\\& & & &\\
	\midrule
	3 & & & &\\& & & &\\
	\midrule
	4 & & & &\\& & & &\\
	\bottomrule
	\end{tabular}
	\end{table}
	
	\begin{itemize}
		\item[(a)] Compare the mean values of the strongest clench in EMG activity between the two forearms. Report the difference between the two forearms as a magnitude (mV or mV-sec) as well as a percentage (\%). Show your calculations.\vspace{6cm}
		\item[(b)] Does the dominant or non-dominant forearm show the highest EMG clench? Explain the physiological basis of your results.\vspace{5cm}
		\item[(c)] List four factors that influence maximum clench strength.\vspace{4cm}
	\end{itemize}
	
	\item[8.]
	\begin{table}[h!]
	\centering
	\caption{Tonus measurements}
	\begin{tabular}[h!]{p{0.1\linewidth}|p{0.2\linewidth}p{0.2\linewidth}|p{0.2\linewidth}p{0.2\linewidth}}
	\toprule
	& Forearm 1 & & Forearm 2 &\\
	Period & Min (mV) & Max (mV) & P-P (mV) & Mean (mV-sec)\\
	\midrule
	1 & & & &\\& & & &\\
	
	\midrule
	2 & & & &\\& & & &\\
	\midrule
	3 & & & &\\& & & &\\
	\bottomrule
	\end{tabular}
	\end{table}
	
	\begin{itemize}
		\item[(a)] Is there a difference in tonus between the two forearms? If so, quantify (and show your work) and explain why.\vspace{6cm}
	\end{itemize}
\end{itemize}
\end{document}
