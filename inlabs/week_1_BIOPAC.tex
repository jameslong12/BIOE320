\documentclass{article}

\usepackage{hyperref}
\hypersetup{
	colorlinks=true,
	linkcolor=blue,
	urlcolor=cyan,}
\usepackage{booktabs}
\usepackage{textgreek}

\input{../structure.tex} % Include the file specifying the document structure and custom commands

%----------------------------------------------------------------------------------------
%	ASSIGNMENT INFORMATION
%----------------------------------------------------------------------------------------

\title{Week 1: Introduction to BIOPAC}
\author{BIOE 320 Systems Physiology Laboratory} 
\date{}
%----------------------------------------------------------------------------------------

\begin{document}
\large
\maketitle

\section*{Objectives}
\begin{enumerate}
	\item To become familiar with the format of data display in the BIOPAC Student Lab data window
	\item To learn how to position data within the window by using software tools and pull down menus
	\item To learn how to select and use correct measurement tools for extracting information from the data
	\item To learn how to use journal to record measurements and write notes
\end{enumerate}

\section*{Background}
The lab is based on the BIOPAC Student Lab System that allows data acquisition and analysis of a variety of physiological signals. \textbf{Hardware} components include electrodes, input transducers, connecting cables, and the acquisition unit (MP30, MP35, or MP36) shown in Fig. \ref{basic_scheme}. The \textbf{software} used in this lab consists of the BIOPAC Student Lab Lessons and the BIOPAC Student Lab PRO pre-installed on your computer.

\begin{figure}[h]
\includegraphics[width=0.5\textwidth]{../images/BIOPAC_1.jpg}
\centering
\caption{Schematic of MP3x acquisition unit and attached transducer}
\label{basic_scheme}
\end{figure}

The lab will introduce you to the BIOPAC system. To complete the in-lab activities, read the instructions carefully and perform each of the steps described in the exercises while answering the questions listed below. Your knowledge and comfort level in manipulating the data obtained using BIOPAC are critical for success in future labs.

\section*{Getting Started}
\begin{enumerate}
	\item From the course website, navigate to \textit{Assignments} and download \textit{SampleData-L07} listed in the week 1 row. Save this file on the Desktop.
	\item Turn on the MP3X acquisition unit by flipping the switch found on the back panel.
		\begin{figure}[h]
		\includegraphics[width=0.8\textwidth]{../images/BIOPAC_2a.jpg}
		\includegraphics[width=0.8\textwidth]{../images/BIOPAC_2b.jpg}
		\centering
		\caption{MP3X acquisition unit - front panel (top) and back panel (bottom) are shown}
		\label{panels}
		\end{figure}
		
	\item Launch the BIOPAC Student Lab Lessons Software (\textit{Start $\rightarrow$ BIOPAC Systems, Inc $\rightarrow$ BIOPAC Student Lab 4.1}).
	\item In the \textit{Choose a Lesson} menu, select \textit{Review Saved Data}. Navigate to the Desktop and select \textit{SampleData-L07}. A window resembling Fig. \ref{sample_data} should appear.
	
		\begin{figure}[h]
		\includegraphics[width=0.8\textwidth]{../images/BIOPAC_21.png}
		\centering
		\caption{L07 data file}
		\label{sample_data}
		\end{figure}
\end{enumerate}

\section*{Data Viewing Tools}
\begin{enumerate}
	\item The software consists of 2 main sections: data window (area with graphs) and journal window (area with text notes). Make sure you can find these on your computer.
	\item In this sample data, the data window contains two series of simultaneously recorded waveforms: an electrocardiogram (ECG, shown in red) and a pulse (shown in blue). These waveforms depend on the physiological parameter measured and will change depending on the lesson recorded.
	\item Select the different markers (found above the marker bar in upside down triangles) and note from the markers what experiments were performed with this data set.
	\item Toggle between channel 1 (ECG) and channel 40 (pulse). Remember that the channel boxes (found above the marker bar on the left) allow you to select different channels in order to concentrate on the desired waveforms.
\end{enumerate}

\section*{Data Analysis Tools: Selection, I-beam, and Zoom Tools}
\begin{enumerate}
	\item Use the selection tool (arrow, located in the top right) to better select the section from 3 to 40 seconds: left click on the x-axis (seconds) to open the prompt seen in Fig. \ref{change_horiz} and change the scale range to display the desired data section.
	
		\begin{figure}[h]
		\includegraphics[width=0.6\textwidth]{../images/BIOPAC_19.png}
		\centering
		\caption{Changing the display for the horizontal scale}
		\label{change_horiz}
		\end{figure}
		
	\item Use the I-beam tool (looks like an I) to highlight different sections of the data. This feature will be expanded upon in the next section.
	\item Use the zoom tool (magnifying glass) to select an area of the data to be enlarged so as to make some analyses easier to perform. Hold down the mouse button and drag the cursor downward and to the right so that the selected area includes approximately six spikes (see Fig. \ref{select_cycles}).
	
		\begin{figure}[h]
		\includegraphics[width=0.8\textwidth]{../images/BIOPAC_22.png}
		\centering
		\caption{Selecting 6 cycles using the zoom tool}
		\label{select_cycles}
		\end{figure}
		
	\item Release the mouse button. Your data should look like Fig. \ref{selected_cycles}.
		\begin{figure}[h]
		\includegraphics[width=0.8\textwidth]{../images/BIOPAC_23.png}
		\centering
		\caption{Selected region using zoom tool}
		\label{selected_cycles}
		\end{figure}	
\end{enumerate}

\section*{Channel Measurement Box}
\begin{enumerate}
	\item Right above the ECG recording channel, you will find the measurement boxes (series of boxes currently listing \textit{None}). The first box indicates the recording channel selected, the second box indicates the function or measurement to be taken, and the third box contains the measurement value (Fig. \ref{meas_box}).
	
		\begin{figure}[h]
		\includegraphics[width=0.4\textwidth]{../images/BIOPAC_20.png}
		\centering
		\caption{Channel measurement boxes}
		\label{meas_box}
		\end{figure}
	\item Set up the following measurement boxes:\begin{itemize}
		\item Channel 1: Delta T
		\item Channel 1: BPM
		\item Channel 1: Max
		\item Channel 1: Min
	\end{itemize}
	\item Using the first cycle (first beat), answer the following questions:
		\begin{enumerate}
			\item What is your measured value for Delta T?
			\item What does Delta T represent?
			\item What is your measured value for BPM?
			\item What does BPM represent?
			\item What are the "max" and "min" for the waveform on ECG?
			\item What do the "max" and "min" values represent?
			\item Try out the options for "area", "P-P", "integral", "stdev", and two additional ones (see Table \ref{table_defs} at the end of this protocol for available options and explanations). Describe a physiological example when each of these measurements may be useful.
		\end{enumerate}
\end{enumerate}

\section*{Overlap/Split}
\begin{enumerate}
	\item These buttons allow you to toggle the display style between an oscilloscope and a chart recorder (see Fig. \ref{overlap_split})
	
		\begin{figure}[h]
		\includegraphics[width=0.8\textwidth]{../images/BIOPAC_23.png}
		\includegraphics[width=0.8\textwidth]{../images/BIOPAC_24.png}
		\centering
		\caption{Display types: split (top) and overlap (bottom)}
		\label{overlap_split}
		\end{figure}
	
	\item Click the overlap button to activate the type of display. Once the signals are overlapped, try to move each signal up and down separately for alignment, using the scroll bar on the right. Select the channel box to select the appropriate channel to adjust.
	\item Return the display type to split.
\end{enumerate}

\section*{Markers}
Markers are used to reference important events (and their associated locations) in the data. There are two types of markers:
\begin{itemize}
	\item \textbf{Append markers}: These appear as a diamond above the marker text box and are blue when active. Append markers are automatically inserted when you begin each new recording segment and are marked with time data. Markers can also be manually entered during recording by pressing F9.
	
	\item \textbf{Event markers}: These appear as inverted triangles below the marker text region and are yellow when active. You may add markers to your data after it has been recorded simply by clicking within the marker region using the selection tool. This new marker will then become the current active marker, and you may type in the marker text box.
\end{itemize}

\begin{enumerate}
	\item How do you insert a marker while data are being recorded?
	\item How do you insert text for a marker?
	\item How do you insert an event marker after data have been recorded? Do this as practice.
\end{enumerate}

\section*{Journal}
The \textit{Review Saved Data} mode incorporates a Journal feature so you can type notes or copy measurements from previously saved data. You can also copy data directly to the Journal. The Journal needs to be the active window for its options to appear.
\begin{enumerate}
	\item To paste a pop-up measurement into the Journal:
		\begin{enumerate}
			\item Select the channel you want to measure by clicking on it with the Selection tool or use the cursor to pick the correct channel number in the boxes to the left of the pop-up measurements.
			\item Choose the appropriate pop-up measurement (e.g. Max, Min).
			\item Use the I-beam tool to select the portion of the wave you are interested in. The pop-up measurement values will update instantly based on the selected area you have chosen with the I-beam tool.
			\item Pull down the \textit{Edit} menu and select \textit{Journal $\rightarrow$ Paste Measurements}. 
		\end{enumerate}
	\end{enumerate}

\section*{Practice}
Using the sample data, display the waveforms contained between 0.2 and 4.4 seconds. Answer the following questions for the pulse data:
		\begin{enumerate}
			\item Maximum values of:
				\begin{enumerate}
					\item First wave?
					\item Second wave?
					\item Third wave?
					\item Fourth wave?
				\end{enumerate}
			\item Frequency of single waveform?
			\item Minimum and maximum amplitudes found between the first and fourth peak?
			\item Interval of time between:
				\begin{enumerate}
					\item First and fourth peak?
					\item Third and fourth peak?
				\end{enumerate}
			\item Briefly describe the function of the following tools:
				\begin{enumerate}
					\item Selection tool
					\item I-Beam tool
					\item Zoom tool
				\end{enumerate}
			\item You have lost track of where you are in the data due to zooming, scrolling, etc. List 2 sequential steps you can take to make sure the complete data file is on the screen.
		\end{enumerate}
\section*{Preparation for Post-Lab}
\begin{enumerate}
	\item Select at least 15-20 seconds of data using the I-beam tool.
	\item Copy the data to the clipboard by going to \textit{Edit $\rightarrow$ Data Window $\rightarrow$ Copy Wave Data}.
	\item Paste these values in a text editor or other software of your choice. Make sure you have data for both channels and email it to yourself.
\end{enumerate} 

\pagebreak
\begin{table}[h!]
	\centering
	\caption{Measurement tools and definitions}
	\label{table_defs}
\begin{tabular}[h!]{p{0.2\linewidth}p{0.75\linewidth}}
\toprule
Measurement tool & Definition\\
\midrule
area & Computes the total area among the waveform and the straight line that is drawn between the endpoints. Expressed in terms of amplitude units $\times$ horizontal units.\\
\midrule
BPM & Calculates the difference in time between the first and last selected points, and divides this value into 60 seconds/minute to extrapolate BPM. If more than one beat is selected, it \textit{will not} calculate the average BPM in the selected area.\\
	& \begin{info}
 	In order to get an accurate BPM value, you must select an area with the I-beam cursor that represents one complete beat-to-beat interval. One way to do this is to select an area that goes from the peak of one cycle's R wave to the peak of the next cycle's R wave.
 	\end{info}\\
\midrule
calculate & Used to perform a calculation using the other measurement results. For example, you can divide the mean pressure by the mean flow\\
	& When \textbf{calculate} is selected, the channel selection box disappears. The result box will read "off" until a calculation is performed, and then it will display the result of the calculation. As you change the selected area, the calculation will update automatically.\\
	& \\
	& To perform a calculation, Ctrl+Click (or right mouse button click) on the \textbf{calculate} measurement type box to generate the "Waveform Arithmetic" dialog. Use the pull-down menus to select the sources and operand.\\
	& \\
	& Measurements are listed by their position in the measurement display grid. Only active, available channels appear in the Source menu. You cannot perform a calculation using the result of another calculation, so calculated measurement channels are not available in the Source menu.\\
	& \\
	& The Constant entry box is activated when you select "Source: K, constant", and it allows you to define the constant value to be used in the calculation. To add units to the calculation result, select the Units entry box and define the unit's abbreviation.\\
\midrule
delta & Computes the difference in amplitude between the last point and the first point of the selected area.\\
\midrule
delta S & Computes the difference in sample points between the end and beginning of the selected area.\\
\midrule
delta T & Computes the difference in time between the end and beginning of the selected area.\\
\midrule
freq & Converts the time segment between the endpoints of the selected area to frequency in cycles/sec. The result is given in units of Hz. It will not calculate the correct frequency if the selected area contains more than one cycle, and you must carefully select the start and end of the cycle.\\
\midrule
integral & Computes the integral value of the data samples between the endpoints of the selected area. The result is expressed in terms of amplitude units $\times$ horizontal units.\\
\midrule
max & Finds the maximum amplitude value within the selected area.\\
\bottomrule
\end{tabular}
\end{table}

\begin{table}[t!]
	\centering
\begin{tabular}[t!]{p{0.2\linewidth}p{0.75\linewidth}}
\toprule
Measurement tool & Definition\\
\midrule
median & Find the median value from the selected area.\\
	& \begin{warn}
		The median calculation is processor-intensive and can take a long time, so only select "median" when you are ready to calculate.
	\end{warn}\\
	\midrule
mean & Computes the mean amplitude value or average of the data samples between the endpoints of the selected area\\
\midrule
min & Finds the minimum amplitude value within the selected area.\\
\midrule
none & Turns off the measurement channel and no result is provided.\\
\midrule
P-P & Finds the maximum value in the selected area and subtracts the minimum value found in the selected area.\\
\midrule
samples & Shows the exact sample number of the selected waveform at the cursor position.\\
\midrule
slope & Uses the endpoints of the selected area to determine the difference in magnitude divided by the time interval. This value is normally expressed in unit change per second (rather than sample points) since high sampling rates can artificially deflate the value of the slope. When an area is selected, the slope measurement computes the slope of the line drawn as a best fit for all selected data points.\\
\midrule
stddev & Computes the standard deviation value of the data samples in the selected range.\\
\midrule
T @ max & Shows the tine of the data point that represents the maximum value of the data samples between the endpoints of the selected area.\\
\midrule
T @ median & Shows the tine of the data point that represents the median value of the data samples between the endpoints of the selected area.\\
	& \begin{warn}
		The median calculation is processor-intensive and can take a long time, so only select "median" when you are ready to calculate.
	\end{warn}\\
\midrule
T @ min & Shows the tine of the data point that represents the minimum value of the data samples between the endpoints of the selected area.\\
\midrule
x-axis: T (time) & Shows the exact time of the selected waveform at the cursor position. If a range of values is selected, then the measurement will indicate the time at the last position of the cursor.\\
\midrule
value & Displays the amplitude value for the channel at the point selected by the I-beam cursor. If a single point is selected, the value is for that point. If an area is selected, the value is the endpoint of the selected area.\\
\bottomrule
\end{tabular}
\end{table}
\end{document}
