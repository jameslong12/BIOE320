\documentclass{article}

\usepackage{hyperref}
\hypersetup{
	colorlinks=true,
	linkcolor=blue,
	urlcolor=cyan,}
\usepackage{booktabs}
\usepackage{textgreek}
\usepackage{gensymb}

\input{../structure.tex} % Include the file specifying the document structure and custom commands

%----------------------------------------------------------------------------------------
%	ASSIGNMENT INFORMATION
%----------------------------------------------------------------------------------------

\title{Week 10 In-Lab}
\author{BIOE 320 Systems Physiology Laboratory} 
\date{}
%----------------------------------------------------------------------------------------

\begin{document}
\large
\maketitle

\section*{Designing an Experiment}

\begin{enumerate}
	\item Specific question:\vspace{2cm}
	\item General research question:\vspace{2cm}
	\item Hypothesis:\vspace{2cm}
	\item Experimental protocol and variables:
	
	\begin{table}[h]
	\centering
	\begin{tabular}[h!]{p{0.3\linewidth}|p{0.65\linewidth}}
	\toprule
	Parameter & Decision\\
	\midrule
	Baseline or reference values & \\\\\\\midrule
	Sample size & \\\\\\\midrule
	Activity your test subject will perform & \\\\\midrule
	Duration of recording and number of repetitions & \\\\\midrule
	Values to analyze & \\\\\\\midrule
	Statistical test to perform & \\\\\\
	\bottomrule
	\end{tabular}
	\end{table}\vspace{3cm}
	
\end{enumerate}


\section*{Experimental Methods}
\subsection*{Data Collection}
\begin{enumerate}
	\item If you need reference values, show a table with the baseline values measured. Otherwise, draw a sample of your data for each group and specify what values you will measure and why.\vspace{7cm}
	\item Write down the instructions that you will provide to the subject before you begin the experiment. Make sure the instruction are concise and repeatable.\pagebreak
	\item Create a table will all experimental values measured. Make sure your table includes the subject number and the parameters you are planning to compare.\pagebreak
\end{enumerate}

\section*{Data Analysis}
\begin{enumerate}
	\item Choose a graph and/or table that summarizes your collected data.\pagebreak
	\item Perform a statistical analysis of your choosing.\vspace{15cm}
	\item State whether or not your hypothesis is supported by the data. Be as specific as possible.\pagebreak
	\item Did your data have any interesting findings? If so, suggest a reason for these interesting findings.\vspace{5cm}
	\item Were there factors not measured that you believe would affect the results? List these factors and justify why they would affect the results.\vspace{5cm}
	\item What are the implications of your results? Relate your findings back to your understanding of physiology. Support your claims with literature if necessary.
\end{enumerate}
\end{document}
