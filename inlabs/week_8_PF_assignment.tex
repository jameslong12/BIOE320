\documentclass{article}

\usepackage{hyperref}
\hypersetup{
	colorlinks=true,
	linkcolor=blue,
	urlcolor=cyan,}
\usepackage{booktabs}
\usepackage{textgreek}

%%%%%%%%%%%%%%%%%%%%%%%%%%%%%%%%%%%%%%%%%
% Lachaise Assignment
% Structure Specification File
% Version 1.0 (26/6/2018)
%
% This template originates from:
% http://www.LaTeXTemplates.com
%
% Authors:
% Marion Lachaise & François Févotte
% Vel (vel@LaTeXTemplates.com)
%
% License:
% CC BY-NC-SA 3.0 (http://creativecommons.org/licenses/by-nc-sa/3.0/)
% 
%%%%%%%%%%%%%%%%%%%%%%%%%%%%%%%%%%%%%%%%%

%----------------------------------------------------------------------------------------
%	PACKAGES AND OTHER DOCUMENT CONFIGURATIONS
%----------------------------------------------------------------------------------------

\usepackage{amsmath,amsfonts,stmaryrd,amssymb} % Math packages

\usepackage{enumerate} % Custom item numbers for enumerations

\usepackage[ruled]{algorithm2e} % Algorithms

\usepackage[framemethod=tikz]{mdframed} % Allows defining custom boxed/framed environments

\usepackage{listings} % File listings, with syntax highlighting
\lstset{
	basicstyle=\ttfamily, % Typeset listings in monospace font
}

%----------------------------------------------------------------------------------------
%	DOCUMENT MARGINS
%----------------------------------------------------------------------------------------

\usepackage{geometry} % Required for adjusting page dimensions and margins

\geometry{
	paper=a4paper, % Paper size, change to letterpaper for US letter size
	top=2.5cm, % Top margin
	bottom=3cm, % Bottom margin
	left=2.5cm, % Left margin
	right=2.5cm, % Right margin
	headheight=14pt, % Header height
	footskip=1.5cm, % Space from the bottom margin to the baseline of the footer
	headsep=1.2cm, % Space from the top margin to the baseline of the header
	%showframe, % Uncomment to show how the type block is set on the page
}

%----------------------------------------------------------------------------------------
%	FONTS
%----------------------------------------------------------------------------------------

\usepackage[utf8]{inputenc} % Required for inputting international characters
\usepackage[T1]{fontenc} % Output font encoding for international characters

\usepackage{XCharter} % Use the XCharter fonts

%----------------------------------------------------------------------------------------
%	COMMAND LINE ENVIRONMENT
%----------------------------------------------------------------------------------------

% Usage:
% \begin{commandline}
%	\begin{verbatim}
%		$ ls
%		
%		Applications	Desktop	...
%	\end{verbatim}
% \end{commandline}

\mdfdefinestyle{commandline}{
	leftmargin=10pt,
	rightmargin=10pt,
	innerleftmargin=15pt,
	middlelinecolor=black!50!white,
	middlelinewidth=2pt,
	frametitlerule=false,
	backgroundcolor=black!5!white,
	frametitle={Command Line},
	frametitlefont={\normalfont\sffamily\color{white}\hspace{-1em}},
	frametitlebackgroundcolor=black!50!white,
	nobreak,
}

% Define a custom environment for command-line snapshots
\newenvironment{commandline}{
	\medskip
	\begin{mdframed}[style=commandline]
}{
	\end{mdframed}
	\medskip
}

%----------------------------------------------------------------------------------------
%	FILE CONTENTS ENVIRONMENT
%----------------------------------------------------------------------------------------

% Usage:
% \begin{file}[optional filename, defaults to "File"]
%	File contents, for example, with a listings environment
% \end{file}

\mdfdefinestyle{file}{
	innertopmargin=1.6\baselineskip,
	innerbottommargin=0.8\baselineskip,
	topline=false, bottomline=false,
	leftline=false, rightline=false,
	leftmargin=2cm,
	rightmargin=2cm,
	singleextra={%
		\draw[fill=black!10!white](P)++(0,-1.2em)rectangle(P-|O);
		\node[anchor=north west]
		at(P-|O){\ttfamily\mdfilename};
		%
		\def\l{3em}
		\draw(O-|P)++(-\l,0)--++(\l,\l)--(P)--(P-|O)--(O)--cycle;
		\draw(O-|P)++(-\l,0)--++(0,\l)--++(\l,0);
	},
	nobreak,
}

% Define a custom environment for file contents
\newenvironment{file}[1][File]{ % Set the default filename to "File"
	\medskip
	\newcommand{\mdfilename}{#1}
	\begin{mdframed}[style=file]
}{
	\end{mdframed}
	\medskip
}

%----------------------------------------------------------------------------------------
%	NUMBERED QUESTIONS ENVIRONMENT
%----------------------------------------------------------------------------------------

% Usage:
% \begin{question}[optional title]
%	Question contents
% \end{question}

\mdfdefinestyle{question}{
	innertopmargin=1.2\baselineskip,
	innerbottommargin=0.8\baselineskip,
	roundcorner=5pt,
	nobreak,
	singleextra={%
		\draw(P-|O)node[xshift=1em,anchor=west,fill=white,draw,rounded corners=5pt]{%
		Question \theQuestion\questionTitle};
	},
}

\newcounter{Question} % Stores the current question number that gets iterated with each new question

% Define a custom environment for numbered questions
\newenvironment{question}[1][\unskip]{
	\bigskip
	\stepcounter{Question}
	\newcommand{\questionTitle}{~#1}
	\begin{mdframed}[style=question]
}{
	\end{mdframed}
	\medskip
}

%----------------------------------------------------------------------------------------
%	WARNING TEXT ENVIRONMENT
%----------------------------------------------------------------------------------------

% Usage:
% \begin{warn}[optional title, defaults to "Warning:"]
%	Contents
% \end{warn}

\mdfdefinestyle{warning}{
	topline=false, bottomline=false,
	leftline=false, rightline=false,
	nobreak,
	singleextra={%
		\draw(P-|O)++(-0.5em,0)node(tmp1){};
		\draw(P-|O)++(0.5em,0)node(tmp2){};
		\fill[black,rotate around={45:(P-|O)}](tmp1)rectangle(tmp2);
		\node at(P-|O){\color{white}\scriptsize\bf !};
		\draw[very thick](P-|O)++(0,-1em)--(O);%--(O-|P);
	}
}

% Define a custom environment for warning text
\newenvironment{warn}[1][Warning:]{ % Set the default warning to "Warning:"
	\medskip
	\begin{mdframed}[style=warning]
		\noindent{\textbf{#1}}
}{
	\end{mdframed}
}

%----------------------------------------------------------------------------------------
%	INFORMATION ENVIRONMENT
%----------------------------------------------------------------------------------------

% Usage:
% \begin{info}[optional title, defaults to "Info:"]
% 	contents
% 	\end{info}

\mdfdefinestyle{info}{%
	topline=false, bottomline=false,
	leftline=false, rightline=false,
	nobreak,
	singleextra={%
		\fill[black](P-|O)circle[radius=0.4em];
		\node at(P-|O){\color{white}\scriptsize\bf i};
		\draw[very thick](P-|O)++(0,-0.8em)--(O);%--(O-|P);
	}
}

% Define a custom environment for information
\newenvironment{info}[1][Info:]{ % Set the default title to "Info:"
	\medskip
	\begin{mdframed}[style=info]
		\noindent{\textbf{#1}}
}{
	\end{mdframed}
}
 % Include the file specifying the document structure and custom commands

%----------------------------------------------------------------------------------------
%	ASSIGNMENT INFORMATION
%----------------------------------------------------------------------------------------

\title{Week 8 In-Lab}
\author{BIOE 320 Systems Physiology Laboratory} 
\date{}
%----------------------------------------------------------------------------------------

\begin{document}
\large
\maketitle

\section*{Experiment 1}
\subsection*{Data Analysis}
\begin{itemize}
	\item[1.] Complete the table. The tidal volume is the peak-to-peak volume measurement of a normal breath cycle. Calculate the Tidal Volume for three different breaths and take an average. Determine IRV, ERV, and VC for one breath.
	
	\begin{table}[h]
	\centering
	\begin{tabular}[h!]{p{0.35\linewidth}|p{0.25\linewidth}}
	\toprule
	Volume title & Volume (L)\\
	\midrule
	Tidal volume (TV) - breath 1 & \\\midrule
	Tidal volume (TV) - breath 2 & \\\midrule
	Tidal volume (TV) - breath 3 & \\\midrule
	Tidal volume (TV) - average & \\\midrule
	Inspiratory Reserve Volume (IRV) & \\\midrule
	Expiratory Reserve Volume (ERV) & \\\midrule
	Vital Capacity (VC) & \\
	\bottomrule
	\end{tabular}
	\end{table}\vspace{0cm}
	
	\item[2.] Calculate the capacities listed in the table assuming a Residual Volume (RV) of 1 L.
	
	\begin{table}[h]
	\centering
	\begin{tabular}[h!]{p{0.35\linewidth}|p{0.25\linewidth}}
	\toprule
	Capacity & Volume (L)\\
	\midrule
	Inspiratory (IC) & \\\midrule
	Expiratory (EC) & \\\midrule
	Functional Residual (FRC) & \\\midrule
	Total Lung (TLC) & \\
	\bottomrule
	\end{tabular}
	\end{table}\vspace{0cm}
	
	\item[3.] How would the volume measurements (TV, IRV, and ERV) change if data were collected after vigorous exercise?\pagebreak
	\item[4.] In this lab, equations for vital capacity (VC) were presented. Using the data provided, determine an equation for VC in the following form:\begin{equation}
		VC = b_0 + b_1x_1 + b_2x_2 + b_3x_3
	\end{equation}
	where $b_n$ is a regression parameter, $x_n$ is a variable (such as height). \textit{Hint: use a program that can perform multiple regression.}\vspace{3cm}
	\item[5.] Qualitatively assess your equation. How do the variables (such as height) affect the VC? Is each of the parameters positively or negatively correlated to VC? What do the magnitude and sign of the regression parameters imply?\vspace{5cm}
\end{itemize}

\section*{Experiment 2}
\subsection*{Data Analysis}
\begin{itemize}
	\item[1.] Estimate the Vital Capacity using a p-p measurement.\vspace{1cm}
	\item[2.] Calculate FEV\textsubscript{1}, FEV\textsubscript{2}, and FEV\textsubscript{3}.
	
	\begin{table}[h]
	\centering
	\begin{tabular}[h!]{p{0.15\linewidth}|p{0.25\linewidth}p{0.25\linewidth}p{0.25\linewidth}}
	\toprule
	FEV\textsubscript{X} & Measured FEV p-p (L) & FEV/VC $\times$ 100 (\%) & Average values\\
	\midrule
	FEV\textsubscript{1} (0-1 sec) & & & 83\%\\\midrule
	FEV\textsubscript{2} (0-2 sec) & & & 94\%\\\midrule
	FEV\textsubscript{3} (0-3 sec) & & & 97\%\\
	\bottomrule
	\end{tabular}
	\end{table}\vspace{0cm}
	
	\item[3.] How do your calculated fractions of air expelled during the three listed time intervals compare with "normal" (average) values?\vspace{2cm}
	\item[4.] Is it possible for a subject to have a vital capacity within normal range, but a value for FEV\textsubscript{1} below normal range? Explain.\vspace{3cm}
	\item[5.] Using the data collected for the Maximal Voluntary Ventilation (MVV) measurements, calculate the respiratory rate (RR) over a 12-second interval. Do so by calculating the number of cycles in the 12-second interval and \textit{not} by highlighting the whole section and selecting Frequency.\vspace{2cm}
	\item[6.] Complete the table with a measurement for each cycle. Complete only for the 12-second interval used above (the table may have more rows than you need). Calculate the average volume per cycle (AVPC).
	
	\begin{table}[h]
	\centering
	\begin{tabular}[h!]{p{0.15\linewidth}|p{0.25\linewidth}}
	\toprule
	Cycle number & p-p volume (L)\\
	\midrule
	1 & \\\midrule
	2 & \\\midrule
	3 & \\\midrule
	4 & \\\midrule
	5 & \\\midrule
	6 & \\\midrule
	7 & \\\midrule
	8 & \\\midrule
	9 & \\\midrule
	10 & \\
	\bottomrule
	\end{tabular}
	\end{table}\vspace{3cm}

	
	\item[7.] Calculate the MVV using the following formula:\begin{equation}
		MVV = AVPC \times RR
	\end{equation}\vspace{2cm}
	
	\item[8.] Bronchodilator drugs open airways and clear mucous. How would these drugs affect the FEV and MVV measurements?\vspace{4cm}
	\item[9.] MVV decreases with age. Why?
\end{itemize}

\end{document}
