\documentclass{article}

\usepackage{hyperref}
\hypersetup{
	colorlinks=true,
	linkcolor=blue,
	urlcolor=cyan,}
\usepackage{booktabs}
\usepackage{textgreek}

\input{../structure.tex} % Include the file specifying the document structure and custom commands

%----------------------------------------------------------------------------------------
%	ASSIGNMENT INFORMATION
%----------------------------------------------------------------------------------------

\title{Week 8 In-Lab}
\author{BIOE 320 Systems Physiology Laboratory} 
\date{}
%----------------------------------------------------------------------------------------

\begin{document}
\large
\maketitle

\section*{Experiment 1}
\subsection*{Data Analysis}
\begin{itemize}
	\item[1.] Complete the table. The tidal volume is the peak-to-peak volume measurement of a normal breath cycle. Calculate the Tidal Volume for three different breaths and take an average. Determine IRV, ERV, and VC for one breath.
	
	\begin{table}[h]
	\centering
	\begin{tabular}[h!]{p{0.35\linewidth}|p{0.25\linewidth}}
	\toprule
	Volume title & Volume (L)\\
	\midrule
	Tidal volume (TV) - breath 1 & \\\midrule
	Tidal volume (TV) - breath 2 & \\\midrule
	Tidal volume (TV) - breath 3 & \\\midrule
	Tidal volume (TV) - average & \\\midrule
	Inspiratory Reserve Volume (IRV) & \\\midrule
	Expiratory Reserve Volume (ERV) & \\\midrule
	Vital Capacity (VC) & \\
	\bottomrule
	\end{tabular}
	\end{table}\vspace{0cm}
	
	\item[2.] Calculate the capacities listed in the table assuming a Residual Volume (RV) of 1 L.
	
	\begin{table}[h]
	\centering
	\begin{tabular}[h!]{p{0.35\linewidth}|p{0.25\linewidth}}
	\toprule
	Capacity & Volume (L)\\
	\midrule
	Inspiratory (IC) & \\\midrule
	Expiratory (EC) & \\\midrule
	Functional Residual (FRC) & \\\midrule
	Total Lung (TLC) & \\
	\bottomrule
	\end{tabular}
	\end{table}\vspace{0cm}
	
	\item[3.] How would the volume measurements (TV, IRV, and ERV) change if data were collected after vigorous exercise?\pagebreak
	\item[4.] In this lab, equations for vital capacity (VC) were presented. Using the data provided, determine an equation for VC in the following form:\begin{equation}
		VC = b_0 + b_1x_1 + b_2x_2 + b_3x_3
	\end{equation}
	where $b_n$ is a regression parameter, $x_n$ is a variable (such as height). \textit{Hint: use a program that can perform multiple regression.}\vspace{3cm}
	\item[5.] Qualitatively assess your equation. How do the variables (such as height) affect the VC? Is each of the parameters positively or negatively correlated to VC? What do the magnitude and sign of the regression parameters imply?\vspace{5cm}
\end{itemize}

\section*{Experiment 2}
\subsection*{Data Analysis}
\begin{itemize}
	\item[1.] Estimate the Vital Capacity using a p-p measurement.\vspace{1cm}
	\item[2.] Calculate FEV\textsubscript{1}, FEV\textsubscript{2}, and FEV\textsubscript{3}.
	
	\begin{table}[h]
	\centering
	\begin{tabular}[h!]{p{0.15\linewidth}|p{0.25\linewidth}p{0.25\linewidth}p{0.25\linewidth}}
	\toprule
	FEV\textsubscript{X} & Measured FEV p-p (L) & FEV/VC $\times$ 100 (\%) & Average values\\
	\midrule
	FEV\textsubscript{1} (0-1 sec) & & & 83\%\\\midrule
	FEV\textsubscript{2} (0-2 sec) & & & 94\%\\\midrule
	FEV\textsubscript{3} (0-3 sec) & & & 97\%\\
	\bottomrule
	\end{tabular}
	\end{table}\vspace{0cm}
	
	\item[3.] How do your calculated fractions of air expelled during the three listed time intervals compare with "normal" (average) values?\vspace{2cm}
	\item[4.] Is it possible for a subject to have a vital capacity within normal range, but a value for FEV\textsubscript{1} below normal range? Explain.\vspace{3cm}
	\item[5.] Using the data collected for the Maximal Voluntary Ventilation (MVV) measurements, calculate the respiratory rate (RR) over a 12-second interval. Do so by calculating the number of cycles in the 12-second interval and \textit{not} by highlighting the whole section and selecting Frequency.\vspace{2cm}
	\item[6.] Complete the table with a measurement for each cycle. Complete only for the 12-second interval used above (the table may have more rows than you need). Calculate the average volume per cycle (AVPC).
	
	\begin{table}[h]
	\centering
	\begin{tabular}[h!]{p{0.15\linewidth}|p{0.25\linewidth}}
	\toprule
	Cycle number & p-p volume (L)\\
	\midrule
	1 & \\\midrule
	2 & \\\midrule
	3 & \\\midrule
	4 & \\\midrule
	5 & \\\midrule
	6 & \\\midrule
	7 & \\\midrule
	8 & \\\midrule
	9 & \\\midrule
	10 & \\
	\bottomrule
	\end{tabular}
	\end{table}\vspace{3cm}

	
	\item[7.] Calculate the MVV using the following formula:\begin{equation}
		MVV = AVPC \times RR
	\end{equation}\vspace{2cm}
	
	\item[8.] Bronchodilator drugs open airways and clear mucous. How would these drugs affect the FEV and MVV measurements?\vspace{4cm}
	\item[9.] MVV decreases with age. Why?
\end{itemize}

\end{document}
