\documentclass{article}

\usepackage{hyperref}
\hypersetup{
	colorlinks=true,
	linkcolor=blue,
	urlcolor=cyan,}
\usepackage{booktabs}
\usepackage{textgreek}

\input{../structure.tex} % Include the file specifying the document structure and custom commands

%----------------------------------------------------------------------------------------
%	ASSIGNMENT INFORMATION
%----------------------------------------------------------------------------------------

\title{Week 9 Pre-Lab}
\author{BIOE 320 Systems Physiology Laboratory} 
\date{}
%----------------------------------------------------------------------------------------

\begin{document}
\maketitle
\large

\textbf{Student Name:}\hfill 	\textbf{Total Grade:\ \ \ \ /10}\vspace{0.5cm}

In this lab, you will be breathing into a 5 L mixing chamber. The mixing chamber will start with "regular" air (20,93 vol\% O\textsubscript{2}, 0.04 vol\% CO\textsubscript{2}, and 79.03 vol\% N\textsubscript{2}). You will inhale air through your nose, which is exposed to "regular" air, and exhale through your mouth into the mixing chamber. As your exhaled air is added into the chamber, an equal volume of air leaves the other end of the chamber so that the system remains at atmospheric pressure. Assume that the chamber is well-mixed.\\

Keep in mind the following points as you sketch plots for this pre-lab:\begin{itemize}
	\item What is the volume  of each exhaled breath as compared to the volume of air in the mixing chamber?
	\item What is a regular breathing rate?
	\item Has there been a gas concentration "plateau" by 2 minutes? In other words, will the concentration of gases in the mixing chamber be the same as the concentration of gases leaving your mouth by 2 minutes?
	\item The x-axis (time) and y-axis (concentration) should be the correct magnitude.
\end{itemize}

\begin{enumerate}
	\item Predict and explain the shape of the graph for percent of O\textsubscript{2} in the mixing chamber vs. time over a 2 minute interval for normal breathing. Repeat for percent of CO\textsubscript{2} in the mixing chamber over time.
	\item Predict and explain the effects of hyperventilation on the percent of O\textsubscript{2} and CO\textsubscript{2} in the mixing chamber vs. time over a 2 minute interval. Qualitatively compare and contrast with predictions from the previous plots.
	\item Predict and explain the effects of exercise on the percent of O\textsubscript{2} and CO\textsubscript{2} in the mixing chamber vs. time over a 2 minute interval. Qualitatively compare and contrast with predictions from the previous plots.
\end{enumerate}
\end{document}