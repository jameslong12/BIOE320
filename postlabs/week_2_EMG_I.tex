\documentclass{article}

\usepackage{hyperref}
\hypersetup{
	colorlinks=true,
	linkcolor=blue,
	urlcolor=cyan,}
\usepackage{booktabs}
\usepackage{textgreek}

\input{../structure.tex} % Include the file specifying the document structure and custom commands

%----------------------------------------------------------------------------------------
%	ASSIGNMENT INFORMATION
%----------------------------------------------------------------------------------------

\title{Week 2 Post-Lab}
\author{BIOE 320 Systems Physiology Laboratory} 
\date{}
%----------------------------------------------------------------------------------------

\begin{document}
\maketitle
\large

\textbf{Student Name:}\hfill 	\textbf{Total Grade:\ \ \ \ /5}\vspace{0.5cm}

\begin{enumerate}
	\item
		\begin{figure}[h]
	\includegraphics[width=0.45\textwidth]{../images/EMG_1_12a.jpg}
	\includegraphics[width=0.45\textwidth]{../images/EMG_1_12b.jpg}
		\centering
		\caption{Taken from Guyton, \textit{Textbook of Medical Physiology} 1986.}
		\label{muscles}
\end{figure}
	
	The figure above shows the isometric contractions of three types of skeletal muscles: an ocular muscle, the gastrocnemius muscle, and the soleus muscle. Note that the duration of the isometric contraction ranges from 20 ms to 200 ms. Why does the duration of the contraction of these muscles vary? Relate the function of the muscle to the contraction duration.
	\item The efficiency of an engine or motor is calculated as the percentage of energy that is converted to work relative to the total energy input to the system. Many engines operate with an efficiency of 60-80\%. In researching muscle, you discover that skeletal and cardiac muscles operate with an efficiency of only 20-20\%.
		\begin{enumerate}
			\item What is the source of energy for muscles in the body?
			\item Where does the rest of the energy go? How is the remainder (75-80\%) of the energy expended? Include a discussion of the role of ATP in energy transformation and storage.
			\item Skeptical of the low efficiency of skeletal muscle, you decide to design some laboratory experiments to measure work and total energy. Describe how you would set up this experiment. What variable(s) would you measure/vary/hold constant? You may design the experiment to use a muscle isolated from the body, a muscle or a group of muscles \textit{in situ}, or the whole body.
		\end{enumerate}
\end{enumerate}
\end{document}