\documentclass{article}

\usepackage{hyperref}
\hypersetup{
	colorlinks=true,
	linkcolor=blue,
	urlcolor=cyan,}
\usepackage{booktabs}
\usepackage{textgreek}

\input{../structure.tex} % Include the file specifying the document structure and custom commands

%----------------------------------------------------------------------------------------
%	ASSIGNMENT INFORMATION
%----------------------------------------------------------------------------------------

\title{Week 8 Post-Lab}
\author{BIOE 320 Systems Physiology Laboratory} 
\date{}
%----------------------------------------------------------------------------------------

\begin{document}
\maketitle
\large

\begin{enumerate}
	\item Calculate FEV\textsubscript{X} from the data you collected.\begin{enumerate}
	\item Using Excel, open your FEV.txt file. The output of the file is in mV (the transducer reports 0.28 mV per 1 L/sec).
	\item Calculate FEV\textsubscript{1}, FEV\textsubscript{2}, and FEV\textsubscript{3} from this data. Fill in the table below. Sure your work, including any code you may use.
		\end{enumerate}
	
		\begin{table}[h]
	\centering
	\caption{Alpha RMS measurements}
	\begin{tabular}[h!]{p{0.15\textwidth}|p{0.15\textwidth}p{0.15\textwidth}}
	\toprule
	FEV\textsubscript{X} & FEV (L) & FEV \%\\
	\midrule
	FEV\textsubscript{1} & &\\ & &\\ \midrule
	FEV\textsubscript{2} & &\\ & &\\ \midrule
	FEV\textsubscript{3} & &\\ & &\\ \bottomrule
	\end{tabular}
	\end{table}
	
	\item What would happen if the pleural space were punctured, opening the space to external atmosphere?
	\item Why is the secretion of surfactant important in the lungs?
\end{enumerate}
\end{document}