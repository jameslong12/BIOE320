\documentclass{article}

\usepackage{hyperref}
\hypersetup{
	colorlinks=true,
	linkcolor=blue,
	urlcolor=cyan,}
\usepackage{booktabs}
\usepackage{textgreek}

\input{../structure.tex} % Include the file specifying the document structure and custom commands

%----------------------------------------------------------------------------------------
%	ASSIGNMENT INFORMATION
%----------------------------------------------------------------------------------------

\title{Week 4 Post-Lab}
\author{BIOE 320 Systems Physiology Laboratory} 
\date{}
%----------------------------------------------------------------------------------------

\begin{document}
\maketitle
\large

\begin{enumerate}
	\item
	\begin{table}[h]
	\centering
	\caption{Alpha RMS measurements}
	\begin{tabular}[h!]{p{0.1\textwidth}|p{0.3\textwidth}p{0.3\textwidth}}
	\toprule
	Sample & Eyes Closed (\textmu V) & Hyperventilation (\textmu V)\\
	\midrule
	1 & &\\ & &\\ \midrule
	2 & &\\ & &\\ \midrule
	3 & &\\ & &\\ \midrule
	4 & &\\ & &\\ \midrule
	5 & &\\ & &\\ \midrule
	6 & &\\ & &\\ \midrule
	7 & &\\ & &\\ \midrule
	8 & &\\ & &\\ \midrule
	9 & &\\ & &\\ \midrule
	10 & &\\ & &\\ \bottomrule
	average & &\\ & &\\ \midrule
	std dev & &\\ & &\\ \bottomrule
	\end{tabular}
	\end{table}
	\large
	
	Fill in the table above with RMS measurements for individual alpha waves within each condition.
	\item Which statistical method is appropriate to determine if the average of the experimental RMS values for the hyperventilation condition is significantly different from the average of the control? Why?
	\item Using the statistical method you suggested above, determine if there is a statistically significant difference between the hyperventilation condition and control. Show your work. You may use Excel,; if you do, print out your data, label your calculations, and attach.
	\item Describe the mathematical/data processing method that allows the deconvolution of the raw EEG signal to the particular waves (e.g. alpha, beta) that are defined by a frequency and amplitude range.
	\item It is unlikely that your EEG data looked exactly like textbook data. One common source of error is 60 Hz noise. Describe this phenomenon and list its sources.
\end{enumerate}
\end{document}