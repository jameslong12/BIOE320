\documentclass{article}

\usepackage{hyperref}
\hypersetup{
	colorlinks=true,
	linkcolor=blue,
	urlcolor=cyan,}
\usepackage{booktabs}
\usepackage{textgreek}

\input{../structure.tex} % Include the file specifying the document structure and custom commands

%----------------------------------------------------------------------------------------
%	ASSIGNMENT INFORMATION
%----------------------------------------------------------------------------------------

\title{Week 9 Post-Lab}
\author{BIOE 320 Systems Physiology Laboratory} 
\date{}
%----------------------------------------------------------------------------------------

\begin{document}
\maketitle
\large

\begin{enumerate}
	\item The BIOPAC software accounted for the conversion of the volume of wet/moist gas at room temperature, pressure and humidity to the volume of dry gas at 0(degree sign) C and 760 mmHg. Develop a formula to calculate a "normalized" dry gas volume (0C, 760 mmHg, no water vapor) from the measured experimental volumes that accounts for variations in temperature, pressure, and water vapor.
	\item On a road trip, you and your 5 friends decided to park at a rest stop and sleep in the van. All windows and doors were closed, making the car airtight. The car's volume is 15 m\textsuperscript{3} with an initial oxygen content of 21\%. Assume sleeping breathing/oxygen consumption rates are the same as recorded normal breathing rates (in reality, it is lower). Humans suffocate at 8\% oxygen air content. Do you need to worry about suffocation or can you sleep peacefully through the night?
	\item You and your parents are talking about the advantages and disadvantages of development in the tropical regions of the world. You know that plants are a key part of the ecosystem, as the process of photosynthesis converts CO\textsubscript{2} to O\textsubscript{2}. Do some research and perform rudimentary calculations, as appropriate, to estimate the following:\begin{itemize}
		\item The average rate of oxygen production from the rainforest.
		\item The average rate of the rainforest destroyed.
		\item The average rate of human population growth.
	\end{itemize} 
	
	Given the current human population, the current acreage of rainforest, and the rates above, are humans in danger of suffocating during your lifetime? Assume no other means of oxygen production and consumption. Assume the average person's breathing/oxygen rate is equal to your normal resting rate. State all other assumptions to support your calculations.
\end{enumerate}
\end{document}