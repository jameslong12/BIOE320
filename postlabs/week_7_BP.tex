\documentclass{article}

\usepackage{hyperref}
\hypersetup{
	colorlinks=true,
	linkcolor=blue,
	urlcolor=cyan,}
\usepackage{booktabs}
\usepackage{textgreek}

\input{../structure.tex} % Include the file specifying the document structure and custom commands

%----------------------------------------------------------------------------------------
%	ASSIGNMENT INFORMATION
%----------------------------------------------------------------------------------------

\title{Week 7 Post-Lab}
\author{BIOE 320 Systems Physiology Laboratory} 
\date{}
%----------------------------------------------------------------------------------------

\begin{document}
\maketitle
\large

\textbf{Student Name:}\hfill 	\textbf{Total Grade:\ \ \ \ /10}\vspace{0.5cm}

\begin{enumerate}
	\item A student collected data to determine the relationship between volumetric blood flow (Q), pressure differential (\textDelta P), and resistance (R). They collected information on the pulmonary circulatory system of ten healthy patients. For each patient, they measured the pressure at two points: the origin of the pulmonary artery and the left atrium. In addition, they determined the volumetric flow rate of blood, Q, and the vascular resistance, R, of each patient. Resistance characterizes the impediment to blood flow through the vessels and is commonly reported in units of PRU, or peripheral resistance units. You may access the student's collected data here.
		\begin{enumerate}
			\item Determine the relationship between Q, \textDelta P, and R. Give the formula relating these three variables.
			\item What are the units of PRU's in terms of mL, sec, and mmHg?
			\item Calculate the average value of R using the values in the chart.
			\item An alternate way to calculate R is to use a graphical method. Determine the value of R by plotting Q and \textDelta P.
			\item Are the values of R from parts (c) and (d) different? Why or why not? Which do you think is a better estimate of R? Why?
		\end{enumerate}
	\item Abnormalities in cardiac output and blood pressure are often related in disease states. One of the most common methods to measure the flow rate of blood is called the indicator-dilution method. Using the indicator-dilution method, a dye is injected into the pulmonary artery and the concentration of this dye is monitored in the radial artery. In a sample patient, 19.3 mg of dye was injected into the pulmonary artery. Using the collect time data, which you can access here, determine the average flow rate in the circulatory system of this patient. Assume that all of the dye stays in the circulatory system.
\end{enumerate}
\end{document}