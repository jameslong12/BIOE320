\documentclass{article}

\usepackage{hyperref}
\hypersetup{
	colorlinks=true,
	linkcolor=blue,
	urlcolor=cyan,}
\usepackage{booktabs}
\usepackage{textgreek}

\input{../structure.tex} % Include the file specifying the document structure and custom commands

%----------------------------------------------------------------------------------------
%	ASSIGNMENT INFORMATION
%----------------------------------------------------------------------------------------

\title{Week 6 Post-Lab}
\author{BIOE 320 Systems Physiology Laboratory} 
\date{}
%----------------------------------------------------------------------------------------

\begin{document}
\maketitle
\large

\textbf{Student Name:}\hfill 	\textbf{Total Grade:\ \ \ \ /10}\vspace{0.5cm}

\begin{enumerate}
	\item A student collected data to determine the relationship between volumetric blood flow (Q), pressure differential (\textDelta P), and resistance (R). They collected information on the pulmonary circulatory system of ten healthy patients. For each patient, they measured the pressure at two points: the origin of the pulmonary artery and the left atrium. In addition, they determined the volumetric flow rate of blood, Q, of each patient. They then calculated vascular resistance, R, which characterizes the impediment to blood flow through the vessels. Resistance is commonly reported in units of PRU, or peripheral resistance units.
	
	\begin{table}[h]
	\centering
	\begin{tabular}[h!]{p{0.18\textwidth}p{0.18\textwidth}p{0.18\textwidth}p{0.18\textwidth}}
	\toprule
	Q (mL/sec) & Pulmonary arterial pressure (mmHg) & Left atrial pressure (mmHg) & R (PRU)\\
	\midrule
	107 & 17 & 2 & 0.140\\\midrule
	55 & 12 & 3 & 0.164\\\midrule
	110 & 14 & 1 & 0.118\\\midrule
	128 & 19 & 3 & 0.125\\\midrule
	125 & 18 & 1 & 0.136\\\midrule
	71 & 13 & 2 & 0.155\\\midrule
	133 & 19 & 1 & 0.135\\\midrule
	85 & 14 & 2 & 0.141\\\midrule
	122 & 18 & 1 & 0.139\\\midrule
	121 & 16 & 2 & 0.116\\\bottomrule
	\end{tabular}
	\end{table}
	
		\begin{enumerate}
			\item Determine the relationship between Q, \textDelta P, and R. Give the formula relating these three variables.
			\item What are the units of PRU's in terms of mL, sec, and mmHg?
			\item Calculate the average value of R using the values in the chart.
			\item An alternate way to calculate R is to use a graphical method. Determine the value of R by plotting Q and \textDelta P.
			\item Are the values of R from parts (c) and (d) different? Why or why not? Which do you think is a better estimate of R? Why?
		\end{enumerate}
	\item Draw and/or explain the pathway through which the body regulates the blood pressure through the juxtaglomerular apparatus in the kidneys. Be sure to include the renin-angiotensin system. {\color{red} prime candidate to cut}
	\item Abnormalities in cardiac output and blood pressure are often related in disease states. One of the most common methods to measure the flow rate of blood is called the indicator-dilution method. Using the indicator-dilution method, a dye is injected into the pulmonary artery and the concentration of this dye is monitored in the radial artery. The table below shows the concentration of the dye in the radial artery over time. 19.3 mg of dye was injected into the pulmonary artery of the patient. Determine the average flow rate in the circulatory system of this patient. Assume that all of the dye stays in the circulatory system. {\color{red} data is in excel}
\end{enumerate}
\end{document}