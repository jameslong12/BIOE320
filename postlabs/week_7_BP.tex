\documentclass{article}

\usepackage{hyperref}
\hypersetup{
	colorlinks=true,
	linkcolor=blue,
	urlcolor=cyan,}
\usepackage{booktabs}
\usepackage{textgreek}

\input{../structure.tex} % Include the file specifying the document structure and custom commands

%----------------------------------------------------------------------------------------
%	ASSIGNMENT INFORMATION
%----------------------------------------------------------------------------------------

\title{Week 7 Post-Lab}
\author{BIOE 320 Systems Physiology Laboratory} 
\date{}
%----------------------------------------------------------------------------------------

\begin{document}
\maketitle
\large

\begin{enumerate}
	\item Abnormalities in cardiac output and blood pressure are often related in disease states. One of the most common methods to measure the flow rate of blood is called the indicator-dilution method. Using the indicator-dilution method, a dye is injected into the pulmonary artery and the concentration of this dye is monitored in the radial artery. In a sample patient, 19.3 mg of dye was injected into the pulmonary artery. Using the collected time data, which you can access \href{https://jameslong12.github.io/BIOE320/Assignments.html}{here}, determine the average flow rate in the circulatory system of this patient. Assume that all of the dye stays in the circulatory system.
\end{enumerate}
\end{document}