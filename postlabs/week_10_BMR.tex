\documentclass{article}

\usepackage{hyperref}
\hypersetup{
	colorlinks=true,
	linkcolor=blue,
	urlcolor=cyan,}
\usepackage{booktabs}
\usepackage{textgreek}

\input{../structure.tex} % Include the file specifying the document structure and custom commands

%----------------------------------------------------------------------------------------
%	ASSIGNMENT INFORMATION
%----------------------------------------------------------------------------------------

\title{Week 10 Post-Lab}
\author{BIOE 320 Systems Physiology Laboratory} 
\date{}
%----------------------------------------------------------------------------------------

\begin{document}
\maketitle
\large

\begin{enumerate}
	\item Create a single plot showing the changes in levels of body glycogen, adipose tissue mass, and muscle mass for an individual fasting for 90 days. You may assume the following:\begin{itemize}
		\item The metabolic rate is constant at 1800 Kcal/day and at least 15\% of this must be derived from glucose.
		\item The starving person initially has 13 kg of pure fat, 4.5 kg of mobilizable protein in muscle, and 0.5 kg of pure glycogen.
		\item No energy is lost in the process of gluconeogenesis (this is a massive oversimplification).
		\item Glycerol does not contribute substantially to the substrate used for gluconeogenesis and acetyl CoA from fat cannot be convert to glucose.
		\item The body uses stores of glycogen first, then begins to metabolize other sources. 
	\end{itemize}
	
	Explain any additional assumptions you make. Do your best to capture the actual amplitude and time course of the changes.
	
	\item Calculate your total energy expenditure (in Kcal) for a typical day. Use \href{https://sites.google.com/site/compendiumofphysicalactivities/}{this website} to help with your estimates.
	\item Assuming a diet of 40\% fat, 15\% protein, and 45\% carbohydrate, how many grams of each must you consume per day to match the demand calculated in the previous question. Is this amount of food close to what you regularly consume?
\end{enumerate}
\end{document}