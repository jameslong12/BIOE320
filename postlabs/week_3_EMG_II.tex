\documentclass{article}

\usepackage{hyperref}
\hypersetup{
	colorlinks=true,
	linkcolor=blue,
	urlcolor=cyan,}
\usepackage{booktabs}
\usepackage{textgreek}

\input{../structure.tex} % Include the file specifying the document structure and custom commands

%----------------------------------------------------------------------------------------
%	ASSIGNMENT INFORMATION
%----------------------------------------------------------------------------------------

\title{Week 3 Post-Lab}
\author{BIOE 320 Systems Physiology Laboratory} 
\date{}
%----------------------------------------------------------------------------------------

\begin{document}
\maketitle
\large

\begin{enumerate}
	\item To determine the absolute and relative refractory periods of a nerve, a student conducted a set of experiments in which a stimulator was configured so that it can give off double pulses (two pulses with equal amplitude in rapid succession) to the nerve. Each pulse was delivered at the maximum stimulation voltage. The voltage of the action potentials was recorded with an oscilloscope. The time delay between the two pulses was initially set at 8 ms. Given this time delay, the amplitude of the two action potentials was the same and was 6.2 mV in magnitude. As the time delay between the pulses decreased, the magnitudes of the second action potential were recorded; the data can be found \href{https://jameslong12.github.io/BIOE320/Assignments.html}{here}.\\
	
	
	Plot the data. Determine and label the absolute refractory period. Determine and label the relative refractory period. Explain your data in light of your understanding of the absolute and relative refractory periods.
	\item During lab, you estimated the nerve conduction velocity along the knee jerk reflex arc. What is the physiological range of nerve conduction velocity? Does your measured value fall in the published range? Name three ways that you could make a more accurate estimate of nerve conduction velocity along this reflex path.
\end{enumerate}
\end{document}